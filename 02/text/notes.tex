\documentclass[11pt, letterpaper]{article}

\usepackage[utf8]{inputenc} % utf-8 encoding
\usepackage[margin=0.75in]{geometry} % page margins
\usepackage{graphicx} % images
\graphicspath{ {images/} } % default images path
\usepackage{float} % use figures
\usepackage{caption} % caption figures
\usepackage{subcaption} % subfigures and subcaptions
\setlength\parindent{0pt} % no paragraph indent
\usepackage{minted} % code samples
\usepackage{amsmath} % math tools

\title{Title}
\author{Stuart Mashaal}
\date{Thursday, November 09, 2017}

\begin{document}

\begin{titlepage}
    \maketitle
\end{titlepage}

\section*{Question 1}
\label{sec:question_1}

\subsection*{The Question}
\label{sub:the_question}

Using the code below, give a sequence of steps that can lead to deadlock. \\

\begin{minipage}{0.5\textwidth}
    \textbf{Process 0}
    \begin{minted}[linenos]{python}
    flag[0] = True
    while (flag[1]): pass
    critical_section()
    flag[0] = False
    \end{minted}
\end{minipage}
\begin{minipage}{0.5\textwidth}
    \textbf{Process 1}
    \begin{minted}[linenos]{python}
    flag[1] = True
    while (flag[0]): pass
    critical_section()
    flag[1] = False
    \end{minted}
\end{minipage}

\subsection*{The Answer}
\label{sub:the_question}

Steps:

\begin{enumerate}
    \item Process 0 runs line 1, setting flag[0] to True
    \item Process 1 runs line 1, setting flag[1] to True
    \item Process 0 runs line 2, enters spin-lock
    \item Process 1 runs line 2, enters spin-lock
\end{enumerate}

Both processes will now loop forever in deadlock if this sequence of steps is followed when the code is run, since they both enter spin locks at the same time.

\pagebreak

\section*{Question 2}
\label{sec:question_2}

\subsection*{The Question}
\label{sub:the_question}

Using the code below, give a sequence of steps, showing the value of the relevant variables at each step, that will cause the code to fail to provide mutual exclusion. \\

\begin{minipage}{0.5\textwidth}
    \textbf{Process 0}
    \begin{minted}[linenos]{python}
    while (turn != 0): pass
    critical_section()
    turn = 1
    non_critical_section()
    \end{minted}
\end{minipage}
\begin{minipage}{0.5\textwidth}
    \textbf{Process 1}
    \begin{minted}[linenos]{python}
    while (turn != 1): pass
    critical_section()
    turn = 0
    non_critical_section()
    \end{minted}
\end{minipage}

\subsection*{The Answer}
\label{sub:the_answer}

Other than the trivial solution where the initial value of the shared variable `turn' is not $\in \{0,1\}$ (in which case neither process ever gets past their spinlock), this code \textit{actually does} provide mutual exclusion.  However, this code does not satisfy one of the other conditions of correctness of a solution to the \textit{critical section problem}, namely that \textbf{a thread blocked from entering a critical section can only be blocked by a thread in the critical section}. \\

Steps:

\begin{enumerate}
    \item Set turn to 0
    \item Process 0 runs lines 1 through 4, entering the non\_criical\_section().  This non\_critical\_section() halts forever
    \item Process 1 runs lines 1 through 4, completing its non\_critical\_section()
\end{enumerate}

Now Process 0 and 1 have both had their turn at running the critical\_section().  However, now that Process 0 is stuck forever inside of its non\_critical\_section(), Process 1 will never be able to re-enter its critical\_section() since `turn' will never get set to 1 again by Process 0.  This means that Process 1 is blocked from entering its critical\_section() \textbf{despite the fact that Process 0 is not in the critical\_section() itself}.

\end{document}
